%\documentclass[conf]{new-aiaa}
%\documentclass[journal]{new-aiaa} for journal papers
\documentclass{article}
\usepackage[utf8]{inputenc}
\usepackage{verbatim}
\usepackage{color,soul}
\usepackage{tabularx}
\usepackage{graphicx}
\usepackage{float}
\usepackage{multirow}
\usepackage[showframe=false]{geometry}
\usepackage{changepage}
\usepackage[T1]{fontenc}
\usepackage{titling}
\predate{}
\postdate{}
\date{}
\setlength{\droptitle}{-10em}  

\newcolumntype{P}[1]{>{\centering\arraybackslash}p{#1}}

\title{Fort Lewis College\\
Physics and Engineering Department\\
Junior Projects - CE315\\
\vspace{.5cm}
Project Definition Document (PDD)\\
\vspace{.5cm}
\textbf{Team Name Here}}


\begin{document}
\vspace{-5cm}
\maketitle
\vspace{-2cm}


\section*{1. Information}

\subsection*{1.1 Course Instructor}
\vspace{-.5cm}
\begin{table}[h!]
\centering
\begin{tabular}{|lc|l}
\hline
Name:  & Dr. Yu Takahashi  \\
Email: & ytakahashi1@fortlewis.edu \\\hline
\end{tabular}
\end{table}


\subsection*{1.2 Team Members}
\vspace{-.5cm}
\begin{table}[h!]
\centering
\begin{tabular}{|l|l|}
\hline
\begin{tabular}[c]{@{}l@{}}Project Manager: Name\\ Email: ???@fortlewis.edu\\ Phone: Phone Number Here\end{tabular} & \begin{tabular}[c]{@{}l@{}}Lead Systems Engineer: Name\\ Email: ???@fortlewis.edu\\ Phone: Phone Number Here\end{tabular} \\ \hline

\begin{tabular}[c]{@{}l@{}}Name: Name\\ Email: ???@fortlewis.edu\\ Phone: Phone Number Here\end{tabular}    & \begin{tabular}[c]{@{}l@{}}Name: Name\\ Email: ???@fortlewis.edu\\ Phone: Phone Number Here\end{tabular}      \\ \hline
\begin{tabular}[c]{@{}l@{}}Name: Name\\ Email: ???@fortlewis.edu\\ Phone: Phone Number Here\end{tabular}        &  \begin{tabular}[c]{@{}l@{}}Name: Name\\ Email: ???@fortlewis.edu\\ Phone: Phone Number Here\end{tabular}       \\ \hline
\begin{tabular}[c]{@{}l@{}}Name: Name\\ Email: ???@fortlewis.edu\\ Phone: Phone Number Here\end{tabular}        & 
\begin{tabular}[c]{@{}l@{}}Name: Name\\ Email: ???@fortlewis.edu\\ Phone: Phone Number Here\end{tabular}   \\\hline

\begin{tabular}[c]{@{}l@{}}Name: Name\\ Email: ???@fortlewis.edu\\ Phone: Phone Number Here\end{tabular}    &  \begin{tabular}[c]{@{}l@{}}Name: Name\\ Email: ???@fortlewis.edu\\ Phone: Phone Number Here\end{tabular}     \\ \hline
\end{tabular}
\end{table}

\pagebreak
\section*{2. Problem or Need}
%Describe the field of application, the problem addressed, and the predicted benefits of a successful project. Pictures or diagrams are encouraged, provided they are annotated or explained clearly. 


\section*{3. Previous Work}
%Place the problem in context with other work. If this is a continuation of a previous project, clearly identify what is novel about your project. Cite references to the engineering literature, popular press, or web sites as appropriate.  Do not cite any proprietary documents or personal communication that is not available in the public domain.    

\section*{4. Specific Objectives}
%Describe specifically what the project must accomplish to satisfy the design problem or need. Define what “success” means for your project, using levels ranging from the absolute minimum that must be accomplished for the project to be considered a success (Level 1) up to the most that the project will plan to accomplish (Level N). Generally, three to four levels are adequate. Do not use “stretch goals” that will be attempted if time permits. Provide an explicit description of the project deliverables to the course and to the customer, including how the system will be tested. 


\section*{5. High Level Functional Requirements}
%Provide an analysis of the high level functional requirements (solution agnostic) to your defined problem, along with explanations on the rationale behind each requirement.  It should also discuss the requirements to be addressed by the team, those that are provided by or acquired from the customer and those that are outside the scope of the project. In many cases, the senior project will address only a portion of a larger system or problem; the functional requirements should clearly distinguish the project elements from others in the larger system.Provide a mission-level Concept of Operations (CONOPS) diagram, showing the specific mission of your project.  (Often, a project addresses only a portion of a larger problem, in which case there is a CONOPS for the larger mission and one for the portion addressed by the project. These two CONOPS will be different). Explain the stages of your mission CONOPS.  It is important that this mission-level CONOPS remain solution agnostic at this stage so as not to limit your solution space.



\section*{6. Critical Project Elements}

%Identify those aspects of the project that are critical to the success defined above, and briefly explain your reasoning. Include technical, logistic, financial, and potential testing/validation constraints.  


\newpage
\section*{7. Team Skills and Interests}
%Describe the areas of expertise and/or interests of the team members on your project, and relate them to the critical project elements identified above.

\begin{table}[H]
\centering
\caption{Team Skills and Interests}

\begin{tabular}{|P{.15\textwidth}|P{.6\textwidth}|P{.15\textwidth}|}
\hline
Team Member & Skills/Interests & CPEs \\ \hline
 &         &       \\ \hline

 &  &  \\ \hline 

 &   &               \\ \hline

 &    &                 \\ \hline

 &   &                \\ \hline

&                 &  
\\ \hline

&      &              \\ \hline


\end{tabular}
\label{tab:skills}
\end{table}

\section*{8. Resources}
%Describe resources beyond team interest/skills needed to address the critical project elements defined above, and identify the source for each. These include specialized equipment, software, facilities, or outside expertise, and any additional financial support needed beyond the $5,000 project funds, along with the source.
\begin{table}[H]
\centering
\caption{Resources}
\begin{tabular}{|P{.3\textwidth}|P{.6\textwidth}|}
\hline
Critical Project Elements & Resource/Source\hspace {7.5cm} \\ \hline
           &       
\\ \hline
 &  \\ \hline

\end{tabular}
\label{tab:Resource}
\end{table}



\newpage
\renewcommand\refname{9. References}
% DO NOT ERASE THIS RENEW COMMAND ^^^
%include at least five references.
\bibliographystyle{unsrt}
\bibliography{sample}

\end{document}
